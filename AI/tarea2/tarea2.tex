%%%%%%%%%%%%%%%%%%%%%%%%%%%%%%%%%%%%%%%%%
% Short Sectioned Assignment
% LaTeX Template
% Version 1.0 (5/5/12)
%
% This template has been downloaded from:
% http://www.LaTeXTemplates.com
%
% Original author:
% Frits Wenneker (http://www.howtotex.com)
%
% License:
% CC BY-NC-SA 3.0 (http://creativecommons.org/licenses/by-nc-sa/3.0/)
%
%%%%%%%%%%%%%%%%%%%%%%%%%%%%%%%%%%%%%%%%%

%----------------------------------------------------------------------------------------
%	PACKAGES AND OTHER DOCUMENT CONFIGURATIONS
%----------------------------------------------------------------------------------------

%% \documentclass[paper=a4, fontsize=11pt]{scrartcl} % A4 paper and 11pt font size
\documentclass[paper=letter, fontsize=12pt]{scrartcl} % Letter paper and 12pt font size

\usepackage[utf8]{inputenc}
\usepackage[T1]{fontenc} % Use 8-bit encoding that has 256 glyphs
\usepackage{fourier} % Use the Adobe Utopia font for the document - comment this line to return to the LaTeX default
\usepackage[english,spanish]{babel} % English language/hyphenation
\usepackage{amsmath,amsfonts,amsthm} % Math packages
\usepackage{listings}
\usepackage{xcolor}

\lstset{literate=
  {á}{{\'a}}1 {é}{{\'e}}1 {í}{{\'i}}1 {ó}{{\'o}}1 {ú}{{\'u}}1
  {Á}{{\'A}}1 {ñ}{{\~n}}1 {É}{{\'E}}1 {Í}{{\'I}}1 {Ó}{{\'O}}1
  {Ú}{{\'U}}1,
  belowcaptionskip=1\baselineskip,
  breaklines=true,
  frame=L,
  xleftmargin=\parindent,
  language=R,
  showstringspaces=false,
  basicstyle=\footnotesize\ttfamily,
  keywordstyle=\bfseries\color{green!40!black},
  commentstyle=\itshape\color{purple!40!black},
  identifierstyle=\color{blue},
  stringstyle=\color{orange},
}

\usepackage{lipsum} % Used for inserting dummy 'Lorem ipsum' text into the template
\usepackage{sectsty} % Allows customizing section commands
\allsectionsfont{\raggedright \textit\normalfont\scshape\emph} % Make all sections centered, the default font and small caps
            
\usepackage{fancyhdr} % Custom headers and footers
\pagestyle{fancyplain} % Makes all pages in the document conform to the custom headers and footers
\fancyhead{} % No page header - if you want one, create it in the same way as the footers below
\fancyfoot[L]{} % Empty left footer
\fancyfoot[C]{} % Empty center footer
\fancyfoot[R]{\thepage} % Page numbering for right footer
\renewcommand{\headrulewidth}{0pt} % Remove header underlines
\renewcommand{\footrulewidth}{0pt} % Remove footer underlines
\setlength{\headheight}{13.6pt} % Customize the height of the header
\setcounter{section}{0}

\numberwithin{equation}{section} % Number equations within sections (i.e. 1.1, 1.2, 2.1, 2.2 instead of 1, 2, 3, 4)
\numberwithin{figure}{section} % Number figures within sections (i.e. 1.1, 1.2, 2.1, 2.2 instead of 1, 2, 3, 4)
\numberwithin{table}{section} % Number tables within sections (i.e. 1.1, 1.2, 2.1, 2.2 instead of 1, 2, 3, 4)

\setlength\parindent{0pt} % Removes all indentation from paragraphs - comment this line for an assignment with lots of text

%----------------------------------------------------------------------------------------
%	TITLE SECTION
%----------------------------------------------------------------------------------------

\newcommand{\horrule}[1]{\rule{\linewidth}{#1}} % Create horizontal rule command with 1 argument of height

\title{
  \vspace*{\fill}
  \begin{center}
    \normalfont \normalsize
    \textsc{Universidad Nacional Autónoma de México\\Facultad de Ciencias} \\ [25pt]
    \horrule{0.5pt} \\[0.4cm] % Thin top horizontal rule
    \huge \textbf{Tarea 2} \\ % The assignment title
    \horrule{2pt} \\[0.5cm] % Thick bottom horizontal rule
    \large \textit{Inteligencia Artificial} \\
  \end{center}
}

\author{Albert Manuel Orozco Camacho} % Your name

\date{\normalsize\today} % Today's date or a custom date
\begin{document}

\begin{titlepage}
  \maketitle
  \thispagestyle{empty}
  \vspace*{\fill}
\end{titlepage}

%----------------------------------------------------------------------------------------
%	PROBLEM 1
%----------------------------------------------------------------------------------------

\section{Agente que convierte imágenes digitalizadas en archivos \LaTeX}

\subsection{Entorno de trabajo}

\begin{itemize}
\item \textbf{Medidas de rendimiento:} Que el texto en \LaTeX   coincida en su totalidad con el de\
  la imagen digitalizada (o en su defecto que se minimice el error de coindidencia por palabra).\
  Que el texto se codifique a texto de \LaTeX   y no a alguna imagen, es decir, que un usuario pueda\
  editar el texto digitalizado dentro de \LaTeX. Que identifique el formato del texto, tablas, gráficos\
  e ilustraciones y todo lo posicione lo más cercano posible a la ``realidad''.
\item \textbf{Entorno:} Conexión con el programa que digitalizó la imagen, con un editor de texto y en\
  general, con un sistema de archivos dentro de un sistema operativo. El programa que digitalizó la imagen\
  debe brindarle al agente una representación de la misma, por ejemplo, una matriz de pixeles.
\item \textbf{Actuadores:} Mecanismo de producción de código \LaTeX   a partir de una representación\
  interna generada por el sensor de imagen. Se trata del programa que transforma de (tal vez) una matriz\
  de pixeles a código \LaTeX.
\item \textbf{Sensores:} Programa que identifica las partes de la imagen que corresponden a texto, a\
  ilustraciones, tablas, ecuaciones y cualquier otro elemento propio de una página de un libro de cálculo.\
  Además, el agente debe de saber interpretar la comunicación que establece con el mismo sistema\
  operativo en donde funciona.
\end{itemize}

\subsection{Seis propiedades de los entornos de trabajo}

\begin{itemize}
\item El agente \textbf{totalmente observa} su entorno pues para identificar las partes de una imagen\
  digitalizada (sensores) lo que se le brinda (matriz de pixeles) le es suficiente para saber qué\
  debe de transformar a \LaTeX. Además, conoce la información del sistema operativo en el que\
  funciona; esto asumiendo que puede adaptarse a diversas computadoras.
\item El agente no sabe más que convertir imágenes de un libro de cálculo en código \LaTeX. Sin embargo,\
  al obtener una imagen, esta no le da información alguna sobre la(s) siguiente(s) imágenes a\
  transformar o si ya acabó de recibir imágenes. Por lo tanto, el entorno es\
  \textbf{estocástico}.
\item Cada imagen a traducir es independiente del modo en el que vienen las demás. Puede decirse\
  que el agente trata cada traducción de manera ``atómica'' ya que no tomará en cuenta el conocimiento\
  previo de traducciones. Por lo tanto, el entorno es \textbf{episódico}. Cabe destacar que esto\
  se preserva incluso si el agente aprendió automáticamente lo que hay que clasificar en una imagen\
  digitalizada. Sin embargo, hay algoritmos como los de aprendizaje por refuerzo que pueden requerir\
  que el agente siga aprendiendo conforme va haciendo traducciones.
\item Normalmente cuando el agente recibe una imagen a traducir, ésta ya no cambia. Realmente considerar\
  cambios en el entorno en esta situación es algo extremo pues prácticamente solo pueden venir de fallas\
  del sistema operativo o la interrrupción del trabajo. Por lo tanto, se trata de un entorno\
  \textbf{estático}.
\item La representación de la imagen digitalizada está dada por una matriz o alguna otra forma discreta\
  que la computadora puede manejar. De ahí a que el entorno sea \textbf{discreto}. Sin embargo, la\
  manera de identificar cada parte de una imagen, ciertamente, asume que la forma de la misma\
  varía continuamente. Dependiendo del algoritmo del sensor es que esto puede llegar a ser tratado\
  como algo continuo pero, en general, se sigue trabajando con valores discretos.
\item A reserva de que se ejecuten varios agentes en paralelo para maximizar la velocidad de traducción\
  de imágenes a \LaTeX, en general, solo se tiene un agente haciendo esta tarea. Por lo tanto, el\
  entorno es de un \textbf{agente individual}, aunque por lo dicho al principio, puede llegar a ser\
  multiagente cooperativo.
\end{itemize}

\section{Damas españolas}

Estoy considerando llamar ``peones'' a cualquier ficha ordinaria y ``dama'' a las que se han\
\textit{coronado}. Éstas últimas son las que poseen dos fichas apiladas.

\begin{itemize}
\item Un estado en este juego es la configuración de la posición de cada ficha en un determinado tiempo\
  y una variable que indica a qué jugador le toca. Esto puede ser modelado por una matriz $M$ de $8 \times 8$ en la que
  \[M[i,j]\ =\ \
  \begin{cases}
    0 & \text{si\ la\ casilla\ i,j\ está\ desocupada}\\
    1 & \text{si\ la\ casilla\ i,j\ está\ ocupada por una dama blanca}\\
    2 & \text{si\ la\ casilla\ i,j\ está\ ocupada por un peón blanco}\\
    3 & \text{si\ la\ casilla\ i,j\ está\ ocupada por una dama negra}\\
    4 & \text{si\ la\ casilla\ i,j\ está\ ocupada por un peón negro}\\
  \end{cases}
  \]
  El espacio de estados es, entonces,
  \[S\ =\ \{\langle M,turn \rangle\ |\ M \in Matriz_{8 \times 8},\ \
  turn \in \{true,false\}\},
  \]
  donde ``turn'' es una variable booleana que indica a qué jugador le toca mover. Se tomará\
  la convención de que si $turn\ =\ true$ entonces juegan las blancas y viceversa.
\item De acuerdo a las reglas, un peón solo se puede mover y capturar una casilla en diagonal\
  y hacia adelante. Las damas hacen casi lo mismo pero en a través de cualquier número de\
  casillas, siempre que se pueda. Decimos que una ficha en $M[i,j]$ es movible a otra coordenada\
  $i',j'$ si de $i,j$ se llega a $i',j'$ en diagonal y
  \[
  \begin{cases}
    llega\ a\ i',j'\ con\ un\ paso\ hacia\ adelante & si\ M[i,j]\ =\ 1\ y\ M[i',j']\ =\ 0\\
    llega\ a\ i',j'\ llendo\ hacia\ adelante\ o\ hacia\ atrás & si\ M[i,j]\ =\ 2\ y\ M[i',j']\ =\ 0\\
  \end{cases}
  \]
  Una ficha en $M[i',j']$ es capturable por otra ficha en $M[i,j]$ si la captura se hace en diagonal y
  \[
  \begin{cases}
    llega\ a\ i',j'\ con\ un\ paso\ hacia\ adelante & si\ M[i,j]\ =\ 1\ y\ M[i',j']\ >\ 0\\
    llega\ a\ i',j'\ llendo\ hacia\ adelante\ o\ hacia\ atrás & si\ M[i,j]\ =\ 2\ y\ M[i',j']\ >\ 0\\
  \end{cases}
  \]
  Una ficha en $M[i,j]$ es coronable si $M[i,j]\ =\ 1$está en el extremo opuesto de a lado. Es decir, si la\
  $i$ es 8 para un peón que empezó en las filas 1, 2 y 3; entonces la $i$ debe ser 0 para\
  peones que empezaron en las filas 8, 7 y 6.\par
  
  Se define el conjunto de acciones como $A\ =\ \{mov,\ capt,\ cor\}$, donde $mov(M,i,j,i',j')$ escribe el\
  contenido de $M[i,j]$ en $M[i',j']$ y $0$ en $M[i,j]$ si esta acción es movible. Además,\
  $capt(M,i,j,i',j')$ escribe el contenido de $M[i,j]$ en $M[i',j']$ y $0$ en $M[i,j]$ si esta acción\
  es capturable. Finalmente, $cor(M,i,j)$ escribe $2$ en $M[i,j]$ si esta acción es coronable.\par

  Obsérvese que todo lo anterior se definió para las fichas blancas pero análogamente, se extiende\
  la definición para fichas negras.\par

  Dado un estado $s_i\ =\ \langle M_i,turn_i \rangle $, se define la función\
  $\gamma:S \times A \to S$ como
  \[
  \gamma(s_{i+1},\ a)\ =\ \
  \begin{cases}
    indefinida & \text{si\ la\ acción\ }a\ \text{mueve piezas\ blancas\ y\ } turn_i\ =\ true\\
    indefinida & \text{si\ la\ acción\ }a\ \text{mueve piezas\ negras\ y\ } turn_i\ =\ false\\
    \langle mov(M_i,k,j,k',j'),!turn_i \rangle & \text{si\ }a\ =\ mov\ \text{para\ algunas} k,k',j,j'\\
    \langle capt(M_i,k,j,k',j'),!turn_i \rangle & \text{si\ }a\ =\ capt\ \text{para\ algunas} k,k',j,j'\\
    \langle cor(M_i,k,j),!turn_i \rangle & \text{si\ }a\ =\ cor\ \text{para\ algunas} k,j\\
  \end{cases}
  \]

  Para el estado inicial, se ignoran los dos casos anteriores que hacen indefinida a la función $\gamma$
  y se establece $turn_0\ =\ true$ si son las blancas las que empiezan o $turn_0\ =\ false$ si son las negras.
  
\item Sea $\Sigma\ = (S,A,E,\gamma)$, donde:
  \begin{itemize}
  \item $S$ es el espacio de estados definido anteriormente
  \item $A$ es el conjunto de acciones definido anteriormente
  \item $E\ =\ \emptyset$
  \item $\gamma$ es la función definida en el inciso anterior
  \end{itemize}

\item Tomando como convención que siempre empiezan las blancas, el estado inicial establece\
  que $turn_0\ =\ true$ y la matriz siguiente:
  \[
  \begin{bmatrix}
    1&0&1&0&1&0&1&0 \\
    0&1&0&1&0&1&0&1 \\
    1&0&1&0&1&0&1&0 \\
    0&0&0&0&0&0&0&0 \\
    0&0&0&0&0&0&0&0 \\
    0&3&0&3&0&3&0&3 \\
    3&0&3&0&3&0&3&0 \\
    0&3&0&3&0&3&0&3 \\
  \end{bmatrix}.
  \]

\item Hay tres escenarios que establecen el término de una partida: que queden dos peones o dos damas por cada\
  bando (tablas), que ya no haya fichas negras (blancas ganan) y que ya no haya fichas blancas\
  (negras ganan). Entonces, la función $g$ que identifica un estado meta es:
  \[
  g(\langle M,a \rangle)\ =\ \
  \begin{cases}
    true & \text{hay\ un\ } 1 \text{\ y\ un\ } 3 \text{\ únicamente}\\
    true & \text{hay\ un\ } 2 \text{\ y\ un\ } 4 \text{\ únicamente}\\
    true & \text{ya\ no\ hay\ } 3\ \text{ni\ } 4\\
    true & \text{ya\ no\ hay\ } 1\ \text{ni\ } 2\\
    false & \text{en\ otro\ caso}\\
  \end{cases}
  \]

\item Dado el estado inicial establecido anteriormente, cada agente (jugador) va realizando la acción que le\
  lleve a llegar lo más pronto posible a algún estado donde el otro agente se quede sin fichas. Es decir,
  cada agente en el $i$-ésimo turno considera la trayectoria $\langle s_i,\ s_{i+1},\ ... ,\ s_{i+m} \rangle$\
  en la que cada estado es alcanzable por el anterior y $s_{i+m}$ es tal que $g(s_{i+m})\ =\ true$ y gana\
  dicho agente, para alguna $m$.
\end{itemize}
 

\end{document}
